
%Abstract, keywords, math subject classification
\begin{abstract}
Non-Local Means is a powerful algorithm for image denoising, yet its high computational complexity limits real-time application. In this paper we provide an implementation and results of this algorithm, while exploring some of its optimizations --- Monte Carlo Non-Local Means, KD-Tree Accelerated, and Hashed NLM. We show that the Monte Carlo optimization utilizes random sampling to accelerate weight calculations without significantly degrading peak signal-to-noise ratio. Our KD-Tree Accelerated NLM implementation improves upon it with a special data structure to efficiently find similar patches instead of random sampling. The Hashed NLM approach (@Petru completeaza aici). We compare the denoising performance and computational efficiency of these implementations on standard test images corrupted with Gaussian noise, first with known sigma values, and then by estimating the noise level using the Fast Fourier Transform.
\newline
\newline
\noindent \textit{Keywords.} Non-Local Means, Denoising, Monte Carlo, Optimization, KD-Trees, FFT, Hashing.
\end{abstract} \maketitle
